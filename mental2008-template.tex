\documentclass[a4paper,fleqn,11pt]{article}
\usepackage{zh_CN-Adobefonts_external} % Simplified Chinese Support using external fonts (./fonts/zh_CN-Adobe/)
\usepackage{fancyhdr}  % 页眉页脚
\usepackage{minted}    % 代码高亮
\usepackage[colorlinks]{hyperref}  % 目录可跳转
\usepackage{indentfirst}
\usepackage{amsmath}
\setlength{\parindent}{0em}
\setlength{\headheight}{15pt}

% 定义页眉页脚
\pagestyle{fancy}
\fancyhf{}
\fancyhead[C]{Algorithm Library by mental2008}
\lfoot{}
\cfoot{\thepage}
\rfoot{}

\author{mental2008}
\title{Algorithm Library}

\begin{document} 
\maketitle % 封面

\newpage % 换页
\tableofcontents % 目录

% 图论 begin
\newpage
\section{图论} % 一级标题

\subsection{网络流}

\subsubsection{Dinic}
\inputminted[breaklines]{c++}{graph/Dinic.cpp}

\subsubsection{ISAP}
\inputminted[breaklines]{c++}{graph/ISAP.cpp}

\subsubsection{经典网络流模型}

"Projects and Instruments"



%\subsection{最小生成树} % 二级标题
%\subsubsection{Kruskal} % 三级标题
%\inputminted[breaklines]{c++}{graph/kruskal.cpp} % 插入代码文件
% 图论 end

%\twocolumn  % 分页显示

% 字符串 begin
\newpage
\section{字符串}
%\subsection{KMP}
%\inputminted[breaklines]{c++}{string/kmp.cpp}
%\subsection{Suffix Automaton}
%\inputminted[breaklines]{c++}{string/suffix-automaton.cpp}
% 字符串 end

\newpage
\section{数据结构}
\subsection{树链剖分}
\inputminted[breaklines]{c++}{DataStructure/树链剖分.cpp}

\subsection{主席树}
\inputminted[breaklines]{c++}{DataStructure/主席树.cpp}

% 数学 begin
\newpage
\section{数学}

\subsection{阶乘逆元}
\inputminted[breaklines]{c++}{math/Fiv.cpp}

\subsection{组合数}
\inputminted[breaklines]{c++}{math/Combination.cpp}

\subsection{卡特兰数}

卡特兰数是一种经典的组合数,经常出现在各种计算中,其前几项为 : 1, 2, 5, 14, 42, 132, 429, 1430, 4862, 16796, 58786, 208012, 742900, 2674440, 9694845, 35357670, 129644790, 477638700, 1767263190, 6564120420, 24466267020, 91482563640, 343059613650, 1289904147324, 4861946401452, ...

卡特兰数满足以下性质:

令$h(0)=1,h(1)=1$,catalan数满足递推式。$h(n)=h(0)*h(n-1)+h(1)*h(n-2)+...+h(n-1)*h(0)$。如果能将公式化为上面这种类型,就是卡特兰数。

通项公式:$h(n) = \tbinom{2n}{n} -\tbinom{2n}{n+1}$,或者$h(n) = \frac{1}{n+1}\tbinom{2n}{n}$。

经典问题:出栈次序、二叉树构成问题、凸多边形的三角形划分。


% 数学 end

% 黑科技 begin
\newpage
\section{黑科技}

\subsection{输入输出外挂}
\subsubsection{简单快读}
\inputminted[breaklines]{c++}{黑科技/QuickIO.cpp}
\subsubsection{真·快速读入}
\inputminted[breaklines]{c++}{黑科技/FastIO.cpp}

\subsection{pbds库}
\subsubsection{Hash}
\inputminted[breaklines]{c++}{黑科技/Hash.cpp}
\subsubsection{可并堆}
\inputminted[breaklines]{c++}{黑科技/Heap.cpp}
\subsubsection{Rope}
\inputminted[breaklines]{c++}{黑科技/Rope.cpp}

\subsection{FWT}
例子:$C_k=\sum_{i|j=k}A_i×B_j,C_k=\sum_{i\&j=k}A_i×B_j,C_k=\sum_{i\oplus j=k}A_i×B_j$
\inputminted[breaklines]{c++}{黑科技/Fwt.cpp}

\subsection{曼哈顿距离与切比雪夫距离}
曼哈顿距离: $d(i,j)=|x1-x2|+|y1-y2|$
切比雪夫距离: $d(i,j)=max(|x1-x2|,|y1-y2|)$

将一个点$(x,y)$的坐标变为$(x+y,x-y)$后
原坐标系中的曼哈顿距离=新坐标系中的切比雪夫距离

将一个点$(x,y)$的坐标变为$(\frac{x+y}{2},\frac{x-y}{2})$后
原坐标系中的切比雪夫距离=新坐标系中的曼哈顿距离

\subsection{Bitset}
\inputminted[breaklines]{c++}{黑科技/Bitset.cpp}

% 黑科技 end

\end{document}
